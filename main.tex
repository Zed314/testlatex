\documentclass[fontsize=18pt]{scrreprt}%23pt
\usepackage[a4paper,margin=0.75in,bottom=0.3in,footskip=2em,includefoot]{geometry}%,landscape
\usepackage{amsmath,amsfonts,mathrsfs,amssymb}%,amsthm}
%\usepackage{pstricks-add}% Pstricks met la page en portrait, bizarrement.
\usepackage{tabularx}
\usepackage{multicol}
% \usepackage{tikz,tkz-tab}
%\usetikzlibrary{arrows,automata,decorations.markings,calc}
% \usepgflibrary{shapes,fpu}
\usepackage{calc}
%\usepackage[english]{babel}
\renewcommand{\thesection}{\Roman{section}}
\renewcommand{\labelenumi}{{\textbf{\arabic{enumi}})}}
\renewcommand{\labelenumii}{{\textbf{\alph{enumii}}.}} %--
\renewcommand{\leq}{\leqslant}
\renewcommand{\geq}{\geqslant}
%EXOS
\newcounter{exercice}
\newcommand{\exo}[1]{% Titre
                     \refstepcounter{exercice}
                     \vspace{1em} \par \noindent
                     \raisebox{-0.7ex}{\textbf{Exercice \no \arabic{exercice}}}
                     \hrulefill\raisebox{-0.7ex}{ \textbf{#1}}
                     \par  \vspace{0.3em} \noindent%
                   }

\usepackage[frenchb]{babel}
\usepackage{pocketmod}
%\usepackage{minilivret}

\usepackage[utf8]{inputenc}
\usepackage[T1]{fontenc}
\usepackage[lining,tabular]{fbb}
\usepackage[scaled=.95,type1]{cabin}
\usepackage[libertine,bigdelims]{newtxmath}

%\usepackage{lipsum}
%\usepackage[math]{blindtext}

\title{Mini-livre d'exercices\\ *** \\ Réussir son entrée en seconde.}
\author{Vincent Pantaloni}
%\date{\includegraphics[width=.85\textwidth]{folding-minibook}}
%\date{Lycée Jean Zay}

%%%%%%%%%%%%%%%%%%%%%%%%%%%%

%%%%%%%%%%%%%%%%%%
\begin{document}
\maketitle

\section{Calculs: fractions, puissances et radicaux.}
%%%%%
\exo{Des fractions}
Calculer sans calculatrice et donner le résultat sous forme de fraction irréductible:
\[A=\frac{-5}{7}+\frac{4}{21};\ B=\frac{5}{72}-\frac{1}{9};\ C=\frac{2}{3}\times\frac{1}{8};\ D=\frac{1}{6}+\frac{1}{6}\times\frac{7}{2};\ E=\frac{\frac12}{3};\ F=\frac{1}{\frac23}\]
%%%%
\exo{Calcul mental$\dots$}
Pour chaque égalité proposée, choisir si elle est vraie ou fausse.\\[1ex]
\renewcommand{\arraystretch}{1.8}
\begin{tabularx}{\linewidth}%
% {|l@{$\quad\Box$\textsc{v} $\Box$\textsc{f}}%
% X|l@{$\quad\Box$\textsc{v} $\Box$\textsc{f}}%
% X|l@{$\quad\Box$\textsc{v} $\Box$\textsc{f}\quad}|}%
{|l@{}%
X|l@{}%
X|l@{}|}%
\hline
$2^3 + 2^2=2^5$&%
&$2^3\times 2^2=2^5$&%
& $\left(2^3\right)^2=2^5$\\\hline%
%
$2^{-3}=0,002$&%
&$2^{-3}\times{2^2}=\frac12$&%$\dfrac{2^3}{2^2}=2$
%&$\left(2^3\right)^2=\left(2^2\right)^3$&%
& $2^5\times 3^5=6^5$\\\hline%
%
$\sqrt{2}+\sqrt{3}=\sqrt{5}$&%
&$\sqrt{2}\times\sqrt{3}=\sqrt{6}$&%
& $\sqrt{3\times 25}=5\sqrt{3}$\\\hline%
%
$\sqrt{(-3)^2}=3$&%
&$-\sqrt{3}^2=3$&%
& $\left(-\sqrt{3}\right)^2=3$\\\hline%
%
$\left(2\sqrt{3}\right)^2=6$&%
&$2\sqrt{3}^2=6$&%
& $\sqrt{3}\sqrt{12}=6$\\\hline%
%
$\frac{\sqrt{50}}{\sqrt2}=5$&%
&$\sqrt{9+16}=7$&%
&$\sqrt{9+16}=5$\\\hline%
%
% $\frac{\sqrt{50}}{\sqrt2}=5$&%
% &$\sqrt{9+16}=7$&%
% &$\sqrt{9+16}=5$\\\hline%
\end{tabularx}
%%%------------------------------------------
%%%%#####################
\newpage
\section{Calcul algébrique}
%%%%#####################
%%%%
\exo{Développements}
Développer et réduire les expressions suivantes, pour tout nombre~$x$:
\vspace{-2ex}
\begin{multicols}{2}
\begin{enumerate}
\item $A(x)=7-2x(5x-3)$
\item $B(x)=(2x-3)(5x-4)$
\item $C(x)=(6+7x)(6-7x)$
\item $D(x)=(x+3)^2$%\left(x+\dfrac12\right)^2$
%\item $E(x)=(4x-1)²$
\end{enumerate}
\end{multicols}
\vspace{-2ex}
%%%%
\exo{Factorisations}
Factoriser les expressions suivantes, pour tout nombre $x$:
\vspace{-2ex}
\begin{multicols}{2}
\begin{enumerate}
\item $x^2+2x$
\item $(x+1)(2x+5)-3(x+1)$
\item $9x^2+3x$
\item $(x-1)^2-16$
% \item $A(x)=x²+2x$
% \item $B(x)=(x+1)(2x+5)-3(x+1)$
% \item $C(x)=9x²+3x$
% \item $D(x)=(x-1)²-16$
\end{enumerate}
\end{multicols}
\vspace{-2ex}
%%%%%%%%%%%%%%%%%
\exo{\'Ecritures littérales}
Faire correspondre à chaque phrase l'expression littérale correcte.\\ Certaines sont déjà proposées: $\dfrac{2+x}{2}$; $2+x$; $2x+3$; $2(x+3)$.\\

\renewcommand{\arraystretch}{1.8}
\begin{tabularx}{\linewidth}{|l|X|}
\hline
La somme de 2 et de $x$. & \\ \hline
Le double de $x$. & \\ \hline
Le carré de $x$.& \\ \hline
La somme de 2 et de la moitié de $x$.& \\ \hline
La moitié de la somme de 2 et $x$.& \\ \hline
La somme de $x$ et du produit de 3 par 2& \\ \hline
Le produit de 2 par la somme de $x$ et de 3.& \\ \hline
La somme du produit de 2 par $x$ et de 3.& \\ \hline
\end{tabularx}

%%%%#####################
\section{\'Equations ou inéquations}
\exo{\'Equations simples}
Résoudre les équations suivantes :
\vspace{-2ex}
\begin{multicols}{3}
\begin{enumerate}
\item $3x+1=-12	$
\item $-2x+5=8$
\item $5x=0$
\item $4-x=7$
\item $ 11x-3=2x+9$
\item ${x}\div{7} = {-7}\div{4} $ %$\dfrac{x}{7} = \dfrac{-7}{4} $
\item $x^2=25 $
\item $x^2=-4 $
\item $4x^2=1 $
\end{enumerate}
\end{multicols}
\vspace{-2ex}
%%%%%
\exo{\'Equations produit nul \& inéquations}
Résoudre les équations et inéquations suivantes.
\vspace{-2ex}
\begin{multicols}{2}
\begin{enumerate}
\item $(x-5)(2x+6)=0$
\item $(2x-8)(5x+4)=0$
%\item $(x-5)(5x+1)-(x-5)(x+2)=0$

\item $-5x\geq 20$
\item $2x+4<5x-7$
%\item $3x+2(5-x)\leq -2x+1$
%\item $3(-x+1)-(2x-4)\geq 5$
\end{enumerate}
\end{multicols}
%\blindmathtrue\blinddocument
\exo{Programme de calcul}
On donne le programme suivant :
Choisir un nombre. Ajouter 3. Calculer le carré du résultat. Soustraire 9. Noter le résultat obtenu.
\begin{enumerate}
\item Montrer que si on choisit le nombre 4, le résultat est~40.
\item On prend maintenant $x$ comme nombre de départ. Exprimer en fonction de $x$, le résultat obtenu. En développant et réduisant l'expression obtenue, on pourra montrer que cette expression est égale à  $x^2+6x$
\item  Quels nombres peut-on choisir pour que le résultat à la fin du programme de calcul soit égal à 0?
\end{enumerate}
%%%%#####################
%%%%#####################
\newpage
\section{Fonctions}
%Exos 6. 1, 3, 4,  7
\exo{Fonctions affines}
Tracer une représentation graphique des fonctions suivantes:
\[f(x)=x-4;	\qquad	g(x)=-2x+3;	\qquad	h(x)=2\]
\vspace{-4ex}
\exo{Image? Antécédent?}
On considère une fonction $f$ définie pour tout nombre $x$ et telle que $f(2)=5$.

On note $\mathscr{C}$ sa courbe représentative dans le plan muni d’un repère.

Répondre en cochant la bonne réponse parmi V (vrai), F (faux) et RI (Renseignements insuffisants pour répondre).
\begin{enumerate}
\item L'image de 5 par la fonction $f$ est 2.\hfill $\Box$\textsc{v} $\Box$\textsc{f} $\Box$\textsc{ri}
\item  L'image de 2 par la fonction $f$ est 5. \hfill $\Box$\textsc{v} $\Box$\textsc{f} $\Box$\textsc{ri}
\item Un antécédent de 5 par la fonction $f$ est 2. \hfill $\Box$\textsc{v} $\Box$\textsc{f} $\Box$\textsc{ri}
\item  2 est le seul antécédent de 5 par la fonction $f$.\hfill $\Box$\textsc{v} $\Box$\textsc{f} $\Box$\textsc{ri}
\item  Un nombre dont l'image est 5 par $f$ est 2. \hfill $\Box$\textsc{v} $\Box$\textsc{f} $\Box$\textsc{ri}
\item  2 a pour image 5 par la fonction $f$. \hfill $\Box$\textsc{v} $\Box$\textsc{f} $\Box$\textsc{ri}
\item Un nombre d'image 7 par $f$ est 2. \hfill $\Box$\textsc{v} $\Box$\textsc{f} $\Box$\textsc{ri}
\item 5 a pour antécédent 2 par la fonction $f$ \hfill $\Box$\textsc{v} $\Box$\textsc{f} $\Box$\textsc{ri}
\item  2 a pour antécédent 5 par la fonction $f$.\hfill $\Box$\textsc{v} $\Box$\textsc{f} $\Box$\textsc{ri}
\item 2 a pour image 7 par la fonction $f$. \hfill $\Box$\textsc{v} $\Box$\textsc{f} $\Box$\textsc{ri}
\item  7 a pour image 2 par la fonction $f$. \hfill $\Box$\textsc{v} $\Box$\textsc{f} $\Box$\textsc{ri}
\item Le point de coordonnées $(2; 5)$ appartient à $\mathscr{C}$. \hfill $\Box$\textsc{v} $\Box$\textsc{f} $\Box$\textsc{ri}
\item  Le point de coordonnées $(5; 2)$ appartient à $\mathscr{C}$.\hfill $\Box$\textsc{v} $\Box$\textsc{f} $\Box$\textsc{ri}
%\item  \hfill $\Box$\textsc{v} $\Box$\textsc{f} $\Box$\textsc{ri}
\end{enumerate}
%%---------------
\exo{}
On considère les fonctions $f$ et $g$ définies pour tout~$x$ par:
%$f(x)=2x-4$ et $g(x)=4x²-5$.
\[f(x)=2x-4\quad\text{et}\quad g(x)=4x^2-5\]
\begin{enumerate}
\item Déterminer l'image de $-3$ par la fonction $f$.
\item Déterminer l'antécédent de 24 par la fonction $f$.
\item Calculer $g(-1)$ et $g(\frac{3}{2})$
\item Déterminer l'image de 4 par la fonction $g$.
\item $\bigstar$ (Pour aller plus loin). Déterminer les antécédents  de 4 par la fonction~$g$.
\end{enumerate}
%%%%
\exo{Lectures graphiques}
Le graphique ci-contre représente la fonction $f$ définie pour tout nombre $x$ par :
\[f(x)=(x-1)^2-3.\]
%\begin{minipage}{\linewidth - 3.2cm}%Une minipage pour l'énoncé
\begin{enumerate}
\item Résolution graphique :\begin{enumerate}
\item Quelles sont les images des nombres $1$ et $-2$ par $f$ ?
\item Quels sont les antécédents par $f$ du nombre $-2$.
\item Existe-t-il un nombre qui admette un et un seul

antécédent par~$f$ ?
%Expliquez votre réponse.
\end{enumerate}
\item Résolution par le calcul : \begin{enumerate}
\item Calculer l'image par $f$ de $0$ et de $2$.

Quel résultat retrouve-t-on ?
\item $\bigstar$ Calculer les antécédents de 6 par~$f$.

Retrouver le résultat par lecture graphique.
\end{enumerate}
\end{enumerate}
%\end{minipage}\hfill%
%\begin{minipage}{3cm}
\clearpage

%{\centering }
% \newrgbcolor{qqwuqq}{0. 0.4 0.}
% \psset{xunit=0.46cm,yunit=0.46cm,algebraic=true,dimen=middle,dotstyle=o,dotsize=3pt 0,linewidth=0.8pt,arrowsize=3pt 2,arrowinset=0.25}
% \begin{pspicture*}(-2.3,-3.2)(4.3,7.)
% \multips(0,-3)(0,1.0){11}{\psline[linestyle=dashed,linecap=1,dash=1.5pt 1.5pt,linewidth=0.4pt,linecolor=gray]{c-c}(-2.3,0)(4.3,0)}
% \multips(-2,0)(1.0,0){7}{\psline[linestyle=dashed,linecap=1,dash=1.5pt 1.5pt,linewidth=0.4pt,linecolor=gray]{c-c}(0,-3.2)(0,7.)}
% \psaxes[labelFontSize=\scriptstyle,xAxis=true,yAxis=true,Dx=1.,Dy=2.,ticksize=-2pt 0,subticks=2]{->}(0,0)(-2.3,-3.2)(4.3,7.)
% \psplot[linewidth=1.2pt,linecolor=qqwuqq,plotpoints=200]{-2.3}{4.3}{(x-1.0)^(2.0)-3.0}
% \begin{scriptsize}
% %\rput[bl](-1.9092,6.198){\qqwuqq{$f$}}
% \end{scriptsize}
% \end{pspicture*}
%\end{minipage}

%%%%#####################
\newpage
\section{Géométrie}
% \exo{Pythagore}
% Le triangle $ABC$ où $AB=$, $BC=$, $AC=$ est-il rectangle?
\exo{Vrai ou Faux?}
\begin{enumerate}
\item Tous les losanges sont des carrés. \hfill $\Box$\textsc{v} $\Box$\textsc{f}
\item Tous les carrés sont des losanges.  \hfill $\Box$\textsc{v} $\Box$\textsc{f}
\item Un quadrilatère dont les diagonales se coupent en leur \\ milieu est un rectangle. \hfill $\Box\textsc{v}\ \Box\textsc{f}$
\item Tout rectangle est un parallélogramme. \hfill $\Box$\textsc{v} $\Box$\textsc{f}
\item  Un quadrilatère qui a quatre côtés égaux est un carré.\hfill $\Box$\textsc{v} $\Box$\textsc{f}
\item Un carré est un rectangle particulier. \hfill $\Box$\textsc{v} $\Box$\textsc{f}
\item Si un quadrilatère a un angle droit et ses diagonales qui \\ se coupent  en leurs milieux, alors c'est un carré. \hfill $\Box$\textsc{v} $\Box$\textsc{f}

\end{enumerate}

\end{document}
